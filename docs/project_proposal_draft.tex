\documentclass[11pt,a4paper]{article}
%%%%%%%%%%%%%%%%%%%%%%%%% Credit %%%%%%%%%%%%%%%%%%%%%%%%

% template ini dibuat oleh martin.manullang@if.itera.ac.id untuk dipergunakan oleh seluruh sivitas akademik itera.

%%%%%%%%%%%%%%%%%%%%%%%%% PACKAGE starts HERE %%%%%%%%%%%%%%%%%%%%%%%%
\usepackage{graphicx}
\usepackage{caption}
\usepackage{microtype}
%\captionsetup[table]{name=Tabel}
%\captionsetup[figure]{name=Gambar}
\usepackage{algorithm}
\usepackage{algpseudocode}
\usepackage{tabulary}
\usepackage{minted}
\usepackage{amsmath}
\usepackage{fancyhdr}

 \usepackage{amssymb}
 \usepackage{amsthm}
\usepackage{placeins}
\usepackage{enumerate}
\usepackage{amsfonts}
\usepackage{graphicx}
\usepackage[all]{xy}
\usepackage{tikz}
\usepackage{verbatim}
\usepackage[left=1cm,right=1cm,top=1.7cm,bottom=1.2cm]{geometry}
\usepackage{hyperref}
\hypersetup{
    colorlinks,
    linkcolor={red!50!black},
    citecolor={blue!50!black},
    urlcolor={blue!80!black}
}
\usepackage{caption}
\usepackage{subcaption}
\usepackage{multirow}
\usepackage{psfrag}
\usepackage[T1]{fontenc}
\usepackage[scaled]{beramono}
% Enable inserting code into the document
\usepackage{listings}
\usepackage{xcolor} 
% custom color & style for listing
\definecolor{codegreen}{rgb}{0,0.6,0}
\definecolor{codegray}{rgb}{0.5,0.5,0.5}
\definecolor{codepurple}{rgb}{0.58,0,0.82}
\definecolor{backcolour}{rgb}{0.95,0.95,0.92}
\definecolor{LightGray}{gray}{0.9}
\lstdefinestyle{mystyle}{
	backgroundcolor=\color{backcolour},   
	commentstyle=\color{green},
	keywordstyle=\color{codegreen},
	numberstyle=\tiny\color{codegray},
	stringstyle=\color{codepurple},
	basicstyle=\ttfamily\footnotesize,
	breakatwhitespace=false,         
	breaklines=true,                 
	captionpos=b,                    
	keepspaces=true,                 
	numbers=left,                    
	numbersep=5pt,                  
	showspaces=false,                
	showstringspaces=false,
	showtabs=false,                  
	tabsize=2
}
\lstset{style=mystyle}
\renewcommand{\lstlistingname}{Kode}
%%%%%%%%%%%%%%%%%%%%%%%%% PACKAGE ends HERE %%%%%%%%%%%%%%%%%%%%%%%%


%%%%%%%%%%%%%%%%%%%%%%%%% Data Diri %%%%%%%%%%%%%%%%%%%%%%%%
\newcommand{\student}{\textbf{Sifar}}
\newcommand{\course}{\textbf{AE6102}}
\newcommand{\assignment}{\textbf{(..)}}

%%%%%%%%%%%%%%%%%%% using theorem style %%%%%%%%%%%%%%%%%%%%
\newtheorem{thm}{Theorem}
\newtheorem{lem}[thm]{Lemma}
\newtheorem{defn}[thm]{Definition}
\newtheorem{exa}[thm]{Example}
\newtheorem{rem}[thm]{Remark}
\newtheorem{coro}[thm]{Corollary}
\newtheorem{quest}{Question}[section]
%%%%%%%%%%%%%%%%%%%%%%%%%%%%%%%%%%%%%%%%
\usepackage{lipsum}%% a garbage package you don't need except to create examples.
\usepackage{fancyhdr}
\pagestyle{fancy}
\lhead{Sifar}
\rhead{ \thepage}
\cfoot{\textbf{AE6102}}
\renewcommand{\headrulewidth}{0.4pt}
\renewcommand{\footrulewidth}{0.4pt}

%%%%%%%%%%%%%%  Shortcut for usual set of numbers  %%%%%%%%%%%

\newcommand{\N}{\mathbb{N}}
\newcommand{\Z}{\mathbb{Z}}
\newcommand{\Q}{\mathbb{Q}}
\newcommand{\R}{\mathbb{R}}
\newcommand{\C}{\mathbb{C}}
\setlength\headheight{14pt}

\lstnewenvironment{TeXlstlisting}[1]{
\lstset{
    frameround=fttt,
    %caption= #2,
    language=#1,
    %label=#3,
    numbers=left,
    breaklines=true,
    delim        = [s][\color{red!50!black}]{$\}\{$},
    moredelim=[s][\color{red!50!black}]{\\[}{\\]},
    moredelim=[s][\color{red!50!black}]{$$\}\{$$},
    keywordstyle=\color{blue}\bfseries, 
    %basicstyle=\ttfamily\color{red},
    numberstyle=\tiny\color{gray},
    commentstyle=\color{green!30!black},
    stringstyle = \color{violet}
    }
    \refstepcounter{Listing}
    {
    \vspace{1.5em}
    }
}{}

%%%%%%%%%%%%%%%%%%%%%%%%%%%%%%%%%%%%%%%%%%%%%%%%%%%%%%%555
\begin{document}
\thispagestyle{empty}
\begin{center}
	\texttt{\Large AE6102 - Parallel Scientific Computing and Visualization}\\
        \vspace*{0.2cm}
        \texttt{\Large Project Proposal}\\
        \vspace*{0.2cm}
	\texttt{\large Spring 2023}
\end{center}
\noindent
\rule{19cm}{0.2cm}\\[0.3cm]
Team Name: \student \hfill Date: \today\\[0.1cm]
\rule{19cm}{0.05cm}
\vspace{0.1cm}

\vspace*{-0.5cm}

\section*{{\LARGE Requirements}}
\begin{itemize}
    \item \texttt{Title}: 3D Visualization and Analysis of Seismic Volumes
    \item \texttt{Participants}
    
    \begin{center}
         \begin{tabular}{||c | c | c||} 
         \hline
         \textbf{Name} & \textbf{Roll Number} & \textbf{Contact} \\ [0.5ex] 
         \hline \hline 
         Adarsh Raj & 190050004 & 190050004@iitb.ac.in \\ 
         \hline 
         Koustav Sen & 190050062 & 190050062@iitb.ac.in \\
         \hline 
         Raja Gond & 190050096 & 190050096@iitb.ac.in \\
         \hline
        \end{tabular}
    \end{center}
    \item \texttt{Abstract}:\\
    
    The project aims to provide a comprehensive and interactive visual representation of subsurface geology by creating three-dimensional images of seismic volumes in \verb!MayaVI! library. The project will facilitate a better understanding of subsurface geology by allowing users to interact with the data in a more intuitive and efficient manner utilizing \verb!TraitsUI! library. Visualization of seismic volumes  is a very crucial component of interpretation workflows, be it to pick salt domes, interpret horizons, identify fault planes, or classify rock facies. 
    
    \item \texttt{Outline}:\\
    
    The project will involve the following steps:

    \begin{itemize}
        \item Collecting seismic data and processing it to generate seismic volumes.
        \item Converting the seismic volumes into 3D models (\verb!numpy! arrays) using a specialized python module \verb!segyio!.
        \item Developing an interactive user interface that allows the user to visualize and manipulate the 3D models, using \verb!TraitsUI!.
        \item Adding functionalities for analysis using \verb!matplotlib! and \verb!mayaVI! to be able to identify fault planes, classification of rock structures, etc.
        \item Adding features such as colouring, slicing, and annotation to enhance the interpretability of the data.
        \item Experiments with popular datasets and demonstration of results of our application corresponding to multiple use cases.
    \end{itemize}

    \item \texttt{Deliverables}:\\
    
    The final project deliverables will include:
    \begin{itemize}
        \item A comprehensive report detailing the methodology and outcomes of the project.
        \item A functional 3D visualization tool allows users to interact with the seismic volumes and view them from different angles and scales, with multiple functionalities for analysis on that seismic volume.
        \item An annotated sample of the 3D model to showcase the features and capabilities of the tool.
    \end{itemize}
    \newpage
    
    \item \texttt{Timeline}:

    \begin{center}
        \begin{tabular}{ || p{0.24\textwidth} | p{0.6\textwidth} || }
        \hline
        \textbf{Date-Date} & \textbf{Planned Progress} \\
        \hline
        12/02/2023 - 26/02/2023 & Project proposal submission \\ [0.4ex]
        & Finalize project based on feedback received during midsem week \\ [0.5ex]
        \hline
        27/02/2023 - 12/03/2023 & Datasets Research, Data Collection, Research on Surface Geology for Analysis Mechanisms, Data Parsing and Transformation into \verb!numpy! 3D models\\ [0.4ex]
        \hline
        13/03/2023 - 26/03/2023 & Models and UI Design, Logics Coding Phase using \verb!MayaVi! and \verb!TraitsUI! \\ [0.4ex]
        \hline
        27/03/2023 - 09/04/2023 & Coding Phase continued, Experimentation on Different factors and tweaks for optimization and better results  \\ [0.4ex]
        \hline
        10/04/2023 - 23/04/2023 & Final report, demo video and optimized code as an open-source GitHub repository. \\ [0.5ex]
        \hline
        \end{tabular}
    \end{center}
    \item \texttt{Git repository}: \url{https://github.com/rajagond/AE6102_sifar} (\textit{Currenly private})
\end{itemize}
\end{document}

